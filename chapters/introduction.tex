\chapter{简介}\label{introduction1}

\section{Modelica概述}\label{overview-of-modelica}

Modelica是一种用于信息物理系统建模的语言,支持由数学方程控制的组件之间的因果联系,以促进从基本原理建模。
它提供了面向对象的结构,便于模型的重用,并且可以方便地用于复杂系统的建模,包括机械、电气、电子、磁性、液压、热、控制、电力或面向过程的子组件。

\section{规范范围}\label{scope-of-the-specification}

Modelica语言的语义是通过一组规则来指定的,这些规则用于将Modelica语言中描述的任何类转换为扁平的Modelica结构。
语义规范应该与Modelica语法一起阅读。

旨在自行模拟的类 (of specialized class \lstinline!model!, \lstinline!class! or \lstinline!block!) 称为 \firstuse{模拟模型}。

扁平的Modelica结构也被定义用于仿真模型以外的其他情况;包括函数(可用于提供算法内容)、包(用作结构机制)和部分模型(用作基础模型)。
这允许在使用这些类构建仿真模型之前验证它们的正确性。

仿真模型有特定的语义限制,以确保模型的完整性;它们允许其平坦的Modelica结构进一步转化为一组微分、代数和离散方程(=平坦混合DAE)。
请注意,满足语义限制并不保证模型可以从初始条件初始化和模拟。

Modelica的设计是为了促进模型的符号转换,特别是通过将基本上每一种Modelica语言构造映射到平面Modelica结构中的方程。
许多Modelica模型,特别是在相关的Modelica标准库中,是更高的指标系统,只有进行符号指标约简,即微分方程和选择适当的变量作为状态,才能合理地模拟。这样得到的方程组可以转化为状态空间形式(至少在局部数值上),即指标为零的混合DAE。
为了允许这种结构分析,如果在翻译过程中参数不能被评估,工具可能会拒绝模拟模型——由于调用外部函数或初始方程/初始算法的 \lstinline!fixed = false! 参数。
接受这样的模型是一个实现质量问题。
Modelica规范没有定义如何模拟模型。
但是,它定义了一组模拟结果应该尽可能满足的方程。

翻译(或扁平化)的关键问题是:
\begin{itemize}
\item
  扩展继承的基类
\item
  基类、局部类和组件的参数化
\item
  从 \lstinline!connect!-equations 生成连接方程
\end{itemize}

扁平混合 DAE 形式包括:
\begin{itemize}
\item
  用适当的基本类型、前缀和属性声明变量,例如 \lstinline!parameter Real v = 5!。
\item
  方程部分中的方程。
\item
  函数调用,调用被视为一组方程,其中包括所有输入变量和所有结果变量(方程的数量=基本结果变量的数量)。
\item
  算法部分,每个部分被视为一组方程,其中涉及到在算法部分中出现的变量(方程的数量=分配的不同变量的数量)。
\item
  在 \lstinline!when! -子句中,每个 \lstinline!when! -子句都被视为一组条件赋值方程,这些方程是子句中出现的变量的函数(方程的数量=分配的不同变量的数量)。
\end{itemize}

因此,扁平混合DAE被看作是一组方程,其中一些方程只是有条件地求值。
模型的初始设置是使用\lstinline!start!-attribute和方程来指定的,它们只在初始化期间成立。

Modelica类也可以包含注释,即正式的注释,它指定类的图形表示(图标和图)、类的文档文本和版本信息。

\section{有关定义}\label{some-definitions}

使用 \cref{document-index}中的文档索引可以找到许多术语的解释。
下面定义了一些重要的术语。

\begin{definition}[Component]\index{component}
Modelica语法中由生产 \lstinline[language=grammar]!组件! 子句定义的元素(基本上是一个变量或一个类的实例)。
\end{definition}

\begin{definition}[Element]\index{element}
类定义、 \lstinline!扩展! 子句或 \lstinline[language=grammar]!组件子句! 在类中声明(基本上是类引用或声明中的组件)。
\end{definition}

\begin{definition}[Flattening]\index{flattening}
将Modelica中描述的模型转换为相应的混合DAE模型,涉及继承基类的扩展、基类、局部类和组件的参数化,以及从 \lstinline!connect!-equations 生成连接方程(基本上,将模型的层次结构映射为一组微分方程、代数方程和离散方程,以及相应的模型变量声明和函数定义)。
\end{definition}

% More terms that would be useful to define here:
% - translation (for phrases such as "during translation")
% - deprecated
% - quality of implementation
% - simulator

\section{Notation}\label{notation}

本节的其余部分将展示本文档中使用的演示示例。

下面的代码清单演示了Modelica代码的语法高亮显示。
需要注意的是,定义代码结构的关键字(如 \lstinline!equation!)、不定义代码结构的关键字(如 \lstinline!connect!),以及由规范定义的具有意义的可识别标识符(如 \lstinline!semiLinear!):
\begin{lstlisting}[language=modelica]
model Example "Example used to illustrate syntax highlighting"
  /* The string above is a class description string, this is a comment. */
  /* Invalid code is typically presented like this: */
  String s = 1.0; // Error: No conversion form Real to String.
  Real x;
equation
  2 * x = semiLinear(time - 0.5, 2, 3);
  /* The annotation below has omitted details represented by an ellipsis: */
  connect(resistor.n, conductor.p) annotation($\ldots$);
end Example;
\end{lstlisting}

依赖于 \lstinline!Integer! 字面值到 \lstinline!Real! 的隐式转换是很常见的,如上面的等式所示(注意在文本中使用了Modelica代码)。

通常将Modelica代码与数学符号混合使用。
例如,\lstinline!average($x$, $y$)! 可以定义为 $\frac{x + y}{2}$.

\begin{definition}[Something]% Do not add this one to the index!
Text defining the meaning of \emph{something}.
\end{definition}

除了上述定义的风格外,还可以在运行的文本中引入新的术语。
% TODO: Switch to \firstuse[---]{dummy} below.  For now, using \willintroduce to avoid risk of accidentally
% creating index entry in the future.
举例来说,\willintroduce{dummy} 是\ldots

\begin{nonnormative}
This is non-normative content that provides some explanation, motivation, and/or additional things to keep in mind.
It has no defining power and may be skipped by readers strictly interested in just the definition of the Modelica language.
\end{nonnormative}

\begin{example}
This is an example, which is a special kind of non-normative content.
Examples often contain a mix of code listings and explanatory text, and this is no exception:
\begin{lstlisting}[language=modelica]
String s = 1.0; // Error: No conversion form Real to String.
\end{lstlisting}
To fix the type mismatch above, the number has to be replaced by a \lstinline!String! expression, such as \lstinline!"1.0"!.
\end{example}

文档中的其他代码清单包括词汇单位和语法结构的规范,两者都使用在~\cref{lex-conventions}中定义的扩展BNF-grammar的元符号。
词汇单位以全部大写字母命名,并通过 `\lstinline[language=grammar]!=!' 标示:
\begin{lstlisting}[language=grammar]
SOME-TOKEN = NON-DIGIT { DIGIT | NON-DIGIT }
\end{lstlisting}
Grammatical structure is recognized by production rules being named with lower-case letters and introduced with the `\lstinline[language=grammar]!:!' sign (also note appearance of the Modelica keyword \lstinline!der!):
\begin{lstlisting}[language=grammar]
differentiated-expression :
    der "(" SOME-TOKEN ")"
    | "(" differentiated-expression "+" differentiated-expression ")"
\end{lstlisting}
