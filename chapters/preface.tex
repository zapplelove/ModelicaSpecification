\chapter*{Preface}\label{preface}
% https://tex.stackexchange.com/questions/45672/adding-numberless-chapters-to-the-table-of-contents
\addcontentsline{toc}{chapter}{\protect\numberline{}Preface}
Modelica是一种免费的面向对象语言,用于对大型、复杂和异构的物理系统进行建模。
从用户的角度来看,模型是由原理图描述的,也称为对象图。
示例如下:
\begin{center}
\includegraphics[width=0.95\textwidth]{diagram_examples}
\end{center}

原理图由连接的部件组成,如电阻器或液压缸。
部件包括\emph{connectors} (通常也被称为\emph{ports}),其描述了交互的可能性,例如,电气引脚、机械法兰或输入信号。 
通过绘制连接器之间的连接线,可以构造物理系统或框图模型。
在内部,组件由另一个原理图定义,或者在``底层'',由Modelica语法中基于方程的模型描述定义。

Modelica语言是一种文本描述,用于定义模型的所有部分,并在称为包的库中构造模型组件。
需要适当的Modelica模拟环境以图形化方式编辑和浏览Modelica模型(通过解释定义Modelica模型的信息),并执行模型模拟和其他分析。
有关此类环境的信息可在\url{https://modelica.org/tools}找到。
基本上,所有Modelica语言元素都映射到微分方程、代数方程和离散方程。
虽然可以合理地定义某些类型的离散偏微分方程,例如,基于有限体积法,并且有Modelica库可以导入有限元程序的结果,但是没有语言元素可以直接描述偏微分方程。

本文档定义了Modelica语言的细节。
本文并不是为了学习Modelica语言。
还有更好的替代方案,如\url{https://modelica.org/publications}上的Modelica参考书籍。
该规范被计算机科学家用来实现Modelica翻译程序,也被那些想要了解特定语言元素的精确细节的建模人员所使用。

直接在章节标题下的文本是章节的非规范介绍。

Modelica语言自1996年开始开发。
本文档描述了Modelica语言的版本\mlsversion{} 。
修订历史可以在\cref{modelica-revision-history}中找到。
